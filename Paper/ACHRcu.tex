%%%%%%%%%%%%%%%%%%%%%%%%%%%%%%%%%%%%%%%%%%%%%%%%%%%%%%%%%%%%%%%%%%%%%%%%%%%%%%%%
%2345678901234567890123456789012345678901234567890123456789012345678901234567890
%        1         2         3         4         5         6         7         8

\documentclass[letterpaper, 10 pt, conference]{ieeeconf}  % Comment this line out
                                                          % if you need a4paper
%\documentclass[a4paper, 10pt, conference]{ieeeconf}      % Use this line for a4
                                                          % paper

\IEEEoverridecommandlockouts                              % This command is only
                                                          % needed if you want to
                                                          % use the \thanks command
\overrideIEEEmargins
% See the \addtolength command later in the file to balance the column lengths
% on the last page of the document



% The following packages can be found on http:\\www.ctan.org
\usepackage{graphics} % for pdf, bitmapped graphics files
%\usepackage{epsfig} % for postscript graphics files
%\usepackage{mathptmx} % assumes new font selection scheme installed
%\usepackage{times} % assumes new font selection scheme installed
\usepackage{amsmath} % assumes amsmath package installed
\usepackage{amssymb}  % assumes amsmath package installed
\usepackage{epstopdf}
\usepackage{epsfig} 
\usepackage{fixltx2e}
\usepackage{pbox}
\usepackage{multirow}% http://ctan.org/pkg/multirow
\usepackage{hhline}% http://ctan.org/pkg/hhline
\usepackage{caption}
\usepackage{hyperref}

%A little hack for the tab command
\newcommand\tab[1][1cm]{\hspace*{#1}}

\title{\LARGE \bf
GPGPU-enabled parallel sampling of large scale metabolic models
}

%\author{ \parbox{3 in}{\centering Huibert Kwakernaak*
%         \thanks{*Use the $\backslash$thanks command to put information here}\\
%         Faculty of Electrical Engineering, Mathematics and Computer Science\\
%         University of Twente\\
%         7500 AE Enschede, The Netherlands\\
%         {\tt\small h.kwakernaak@autsubmit.com}}
%         \hspace*{ 0.5 in}
%         \parbox{3 in}{ \centering Pradeep Misra**
%         \thanks{**The footnote marks may be inserted manually}\\
%        Department of Electrical Engineering \\
%         Wright State University\\
%         Dayton, OH 45435, USA\\
%         {\tt\small pmisra@cs.wright.edu}}
%}

\author{Marouen Ben Guebila$^{1}$ and Ines Thiele$^{1*}$% <-this % stops a space
\thanks{$^{1}$Molecular Systems Physiology group at the Luxembourg Centre for Systems Biomedicine,
        University of Luxembourg, Campus Belval.
        }%
\thanks{*corresponding author: {\tt\small ines.thiele@uni.lu}}
}


\begin{document}



\maketitle
\thispagestyle{empty}
\pagestyle{empty}


%%%%%%%%%%%%%%%%%%%%%%%%%%%%%%%%%%%%%%%%%%%%%%%%%%%%%%%%%%%%%%%%%%%%%%%%%%%%%%%%
\begin{abstract}



\end{abstract}


%%%%%%%%%%%%%%%%%%%%%%%%%%%%%%%%%%%%%%%%%%%%%%%%%%%%%%%%%%%%%%%%%%%%%%%%%%%%%%%%
\section{INTRODUCTION}




\addtolength{\textheight}{-12cm}   % This command serves to balance the column lengths
                                  % on the last page of the document manually. It shortens
                                  % the textheight of the last page by a suitable amount.
                                  % This command does not take effect until the next page
                                  % so it should come on the page before the last. Make
                                  % sure that you do not shorten the textheight too much.

%%%%%%%%%%%%%%%%%%%%%%%%%%%%%%%%%%%%%%%%%%%%%%%%%%%%%%%%%%%%%%%%%%%%%%%%%%%%%%%%



%%%%%%%%%%%%%%%%%%%%%%%%%%%%%%%%%%%%%%%%%%%%%%%%%%%%%%%%%%%%%%%%%%%%%%%%%%%%%%%%



%%%%%%%%%%%%%%%%%%%%%%%%%%%%%%%%%%%%%%%%%%%%%%%%%%%%%%%%%%%%%%%%%%%%%%%%%%%%%%%%

\section{Material and methods}

\section{Results}
\begin{tabular}{|l|c|r|c|c|c|}
  \hline
  Model & Points & Steps & Intel Xeon (3.5 Ghz) & Quadro K600 (0.87 Ghz) & Tesla M2090 \\
  \hline
   Ecoli core& 1000& 1000& 42 & 1.9 & 1.2\\
   Ecoli core& 5000& 1000& 208 & 8.3 & 1.88\\
   Ecoli core& 10000& 1000& 420 & 17.8 & 3.33\\
   P Putida  & 1000& 1000& 103 & 54 & 32\\
   P Putida  & 5000& 1000& 516 &  223 & 59\\
   P Putida  & 10000& 1000& 1081 & 440 & 111\\
   Recon2  & 1000& 1000&  & & \\
   Recon2  & 5000& 1000&  & & \\
   Recon2  & 10000& 1000&  & & \\
  \hline
\end{tabular}
\section*{ACKNOWLEDGMENT}

 The experiments presented in this paper were carried out
using the HPC facilities of the University of Luxembourg~\cite{VBCG_HPCS14} 
{\small -- see \url{http://hpc.uni.lu}}.



%%%%%%%%%%%%%%%%%%%%%%%%%%%%%%%%%%%%%%%%%%%%%%%%%%%%%%%%%%%%%%%%%%%%%%%%%%%%%%%%

\bibliography{refExtr} 
\bibliographystyle{ieeetr}



\end{document}